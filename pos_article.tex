% Please make sure you insert your
% data according to the instructions
% in PoSauthmanual.pdf available at http://pos.sissa.it/POSauthors.html
\documentclass{PoS}

\title{Overview of Machine Learning and Big data tools at HEP experiments}

\ShortTitle{ML in HEP}

\author{\speaker{A. Castaneda}\thanks{A footnote may follow.}\\
        Universidad de Sonora\\
        E-mail: \email{castaned@cern.ch}}

%\author{Another Author\\
%        Affiliation\\
%        E-mail: \email{...}}

\abstract{}

\FullConference{Large Hadron Collider LHCP\\
                May, 2019\\
                Puebla, Mexico}


\begin{document}

\section{Introduction}


There is a growing interest on the implementation of machine learning and big data tools to be used by modern particle physics
experiments, in particular those associated with the Large Hadron Collider (LHC) at the CERN laboratory. The LHC has sucessfully
collected data from 2010 to 2018, having the discovery of the Higgs boson among its greater achievements. The LHC has started
a manteinance and upgrade period towards its phase-2, also known, High Luminosity era in which the collision frecuency will increase one
order of magnitude compared to the current operation. This will allow scientist to fully exploit the detector capabilities and increase
the probabilities for the discovery of new physics phenomena.  New analysis and data processing methods need to be developed to cope with
this increase on luminosity. In the past Multivariate techniques have been used to disentangle between signal and background processes,
nevertheless with the development of deep learning it seems possible to optimize several tasks as for instance those related to particle identification, reconstruction and triggering.
Also the use of rather new architectures such as the Generative Adversarial networks (GANs) seems promising to reduce the simulation time.


\section{Jet classification}

Jet classification is one of the most important tasks in particle physics. Identifying the nature of a jet through its internal structure
has broad applications to searches for new physics and standard model measurements at the LHC.  A jet could be initiated by a quark or a gluon,
the main difference is that gluon jets tend to have more constituents and a broader radiation pattern than quark jets, traditional approach uses
a set of key observables to achieve such separation, however a rather new technique that uses state-of-the-art image classification techniques has
proven to have similar or better performance. Jets are identified by the energy deposition of their constituiend particles in the calorimeters.
this energy deposition could be interpreted as an 
image as the one shown in Figure~\ref{fig:jet_images}.  A deep neural network arquitecture known as Convolutional Neural Network (CNN) can be used
to train a model that recognize whether a jet is originated by a quark or gluon, a CNN uses an array of layers that capture the structure of the image,
the first layers act as filters in which there is a reduction in the dimensionality of the image, the last layer is a fully connected layer that receive this
information and classify the jet accordingly, this arquitecture is shown in Figure~\ref{fig:cnn_schema}. The CNN was trained and the performance evaulated
using Monte Carlo simulation samples as described in~\cite{jet_cnn}.  The performance is shown in Figure~\ref{fig:cnn_schema}

\begin{figure}
\begin{center}
  \includegraphics[width=1.8in]{/home/alfredo/Pictures/ATLAS_quarkjet.png}
  \includegraphics[width=1.95in]{/home/alfredo/Pictures/ATLAS_gluonjet.png}
  \caption{Quark jet image (left) and gluon jet image (right) produced with Monte Carlo simulation. Quark-jets are more collimated than gluon ones}
  \label{fig:jet_images}
\end{center}
\end{figure}


\begin{figure}
\begin{center}
  \includegraphics[width=4.4in]{/home/alfredo/Pictures/CNN_for_jets.png}
  \caption{Illustration of a deep convolutional neural network arquitecture.}
  \label{fig:cnn_schema}
\end{center}
\end{figure}






\section{GANs for simulation}

The production of Monte Carlo simulated samples is one of the crucial tasks in High Energy Physics. Currently this production represents
more than 50\% of the computing load of the LHC computing (GRID). Detailed simulation is needed to optimize the event selection for Standard
Model and new physics searches, evaluate the performance of the detector systems. In the case of rare processes large samples are needed in
order to reduce the statistical uncertainties. The most computationally expensive step in the simulation pipeline of a typical experiment at the
LHC is the detailed modeling of the full complexity of physics processes such as the evolution of particle showers inside calorimeters. 
Generative Adversarial Networks (GANs) are a type of deep neural network composed of two nets, the generator and the discriminator.  GANs are
trained with a known dataset, the generator usually starts with a seeded randomized input that is evaluated by the discriminator. Using a
backpropagation algorithm the generator is able to learn from the training dataset until it reaches a point the discriminator is not able do
distinguish between the true information and the one coming from the generator.  This principle is used for the simulation of showers in
high energy physics.  The true showers are generated with a full simulation using GEANT4 package and compared with the output of a CaloGAN,
as shown in Figure~\ref{fig:jet_images}.


\begin{figure}
\begin{center}
  \includegraphics[width=2.2in]{/home/alfredo/Pictures/GAN_images.png}
  \includegraphics[width=1.6in]{/home/alfredo/Pictures/shower_gans.png}
  \caption{Quark jet image (left) and gluon jet image (right) produced with Monte Carlo simulation. Quark-jets are more collimated than gluon ones}
  \label{fig:jet_images}
\end{center}
\end{figure}


\section{Big data tools}

The LHC collide protons with a frequency of 40MHz, however the storage of this amount of data is unfeasible and is strongly reduced with a
set of triggers applied at online and offline levels. 





\begin{thebibliography}{99}
\bibitem{jet_cnn}  \emph{Quark versus Gluon Jet Tagging Using Jet Images with the ATLAS Detector}, CERN, ATLAS Collaboration, "ATL-PHYS-PUB-2017-017",Jul 2017, ATL-PHYS-PUB-2017-017, https://cds.cern.ch/record/2275641
\bibitem{gans}  \emph{CaloGAN: Simulating 3D high energy particle showers in multilayer electromagnetic calorimeters with generative adversarial networks},
  PhysRevD.97.014021,Paganini, Michela and de Oliveira, Luke and Nachman, Benjamin, Phys. Rev. D, volume 97, issue = 1, pages = 014021, numpages = 12, year = 2018,
  month = {Jan}, doi = {10.1103/PhysRevD.97.014021}
\end{thebibliography}



\end{document}
