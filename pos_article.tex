% Please make sure you insert your
% data according to the instructions
% in PoSauthmanual.pdf available at http://pos.sissa.it/POSauthors.html
\documentclass{PoS}

\title{Overview of Machine Learning and Big data tools at HEP experiments}

\ShortTitle{ML in HEP}

\author{\speaker{A. Castaneda}\thanks{A footnote may follow.}\\
        Universidad de Sonora\\
        E-mail: \email{castaned@cern.ch}}

%\author{Another Author\\
%        Affiliation\\
%        E-mail: \email{...}}

\abstract{}

\FullConference{Conference title\\
                Date\\
                Location}


\begin{document}

\section{Introduction}


The interest and development of machine learning and big data tools in the framework of particle physics experiments
is being growing during the most recent years. With the high luminosity era ahead it is clear that the development of new
analysis techniques and data processing will be fundamental to exploite the detector capabilities and increase the probabilities
for the discovery of new physics phenomena. Multivariate analysis  has been used in the past to disentangle between signal and
background processe, nevertheless with the development of deep learning it seems possible to optimize several tasks in HEP as for instance
particle identificacion, reconstruction, triggering.   The use of complicate Generative Adversarial networks seems promising to
reduce the simulation time and the integration of machine learning models at the hardware level may be crucial to process in a smart
way the signals received during the messy environment in high luminosity 


\section{PID, reconstruction and triggering}

\section{GANs for simulation}

\section{big data tools}

\begin{thebibliography}{99}
\bibitem{...}
....

\end{thebibliography}

\end{document}
