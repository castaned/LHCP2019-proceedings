% Please make sure you insert your
% data according to the instructions
% in PoSauthmanual.pdf available at http://pos.sissa.it/POSauthors.html
\documentclass{PoS}

\title{Overview of Machine Learning and Big data tools at HEP experiments}

\ShortTitle{ML in HEP}

\author{\speaker{A. Castaneda}\thanks{A footnote may follow.}\\
        Universidad de Sonora\\
        E-mail: \email{castaned@cern.ch}}

%\author{Another Author\\
%        Affiliation\\
%        E-mail: \email{...}}

\abstract{}

\FullConference{Large Hadron Collider LHCP\\
                May, 2019\\
                Puebla, Mexico}


\begin{document}

\section{Introduction}


The interest and development of machine learning and big data tools to be used in the framework of particle physics experiments
has being growing during the most recent years. The Large Hadron Collider (LHC) has started the work towards the so called
High Luminosity era in which the collision frequency will increase by one order of magnitude.  This will allow scientist to
explode the detector capabilities and increase the probabilities
for the discovery of new physics phenomena. Multivariate analysis  has been used in the past to disentangle between signal and
background processes, nevertheless with the development of deep learning it seems possible to optimize several tasks in HEP as for instance
particle identification, reconstruction, triggering.   The use of complicate Generative Adversarial networks seems promising to
reduce the simulation time and the integration of machine learning models at the hardware level may be crucial to process in a smart
way the signals received during the messy environment in high luminosity 


\section{Particle Identification, Reconstruction and Triggering}


Jet classification is one of the most important tasks in experimental particle physics. Jets could be originated from quarks or gluons
in the case of quarks a jet could be further classified as if it was originated from a heavy (b,c,t) or light jet (u,d,s). 

Jets are identified by the deposition of energy in the electromagnetic or hadronic calorimeter, this energy deposition could be interpreted in a
2D image as the one shown in Figure~\ref{}.  A first application of deep learning is the discrimination between a quark-initiated jet versus
a gluon initiated one, this differentiation is important because it reveals the internal structure of a jet, experimentally is observed that
gluon jets tend to have more constituents and broader radiation pattern than quark jets, this difference is exploited to train a Convolutional
Neural Network (CNN) based on a series of layers that allow to train a model that differentiate between these two categories.

These
differentiation is important as for instance many of the rare process in standard model and some of the new physics theories are
dominated for a production of heavy quark jet, as in the case of pairs of tops quarks and supersymmetry models.





\section{GANs for simulation}

Production of Monte Carlo simulation is one of the crucial tasks in High Energy Physics. It represents more than 50\% of the computing load of
the GRID system.  Detailed simulation is needed to optimize the event selection for Standard Model and new physics searches, evaluate the
performance of the detector systems. In the case of rare processes large samples are needed in order to reduce the statistical uncertainties.

Generative Adversarial Networks (GANs) are a type of deep neural network composed of two nets, one of the networks is the generator and the other






\section{Big data tools}

The LHC collide protons with a frequency of 40MHz, however the storage of this amount of data is unfeasible and is strongly reduced with a
set of triggers applied at online and offline levels. 





\begin{thebibliography}{99}
\bibitem{...}


\end{thebibliography}

\end{document}
